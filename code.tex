; matrix multiplication program for grad-pl
; concurrent programming languages group
; program written by: Derek Gaffney
; last update: April 21, 2017
(ns matrix-multiplication.core
  (:gen-class))

;m is number of rows for matrix a
;r is number of columns for matrix a
;r is number of rows for matrix b
;n is number of columns for matrix b
(defn define-constants
  []
  ; for square matrices m == r == n
  (def ^:const m 1500)
  (def ^:const r 1500)
  (def ^:const n 1500)
  (def ^:const max-num-size 10)) ;max size of a number in a matrix

; define how many cores the computer has
; work load among cores will be optimized based on this number
(def ^:const num-of-cores 8)

;given a size of how many numbers should be in the vector,
; returns a vector of that size filled with random numbers
(defn rand-num-vector-of-size
  [size]
  (into [] (take size (repeatedly #(rand-int max-num-size)))))

;given number of vectors v and size of those vectors s
; create a vector of vectors to simulate a matrix
(defn gen-v-vectors-size-s
  [v s]
  (into [] (take v (repeatedly #(rand-num-vector-of-size s)))))

;define matricies a and b with sizes defined by
; constants m r and n
;a is defined as row vector
;b is defined as column vector
(defn define-matrices
  []
  (def a (gen-v-vectors-size-s m r))
  (def b (gen-v-vectors-size-s n r)))

;make pair-vectors that pair each vector from a
; with its respective vector in b which will need
; to have the dot product taken of
(defn make-multiple-pair-vectors
  []
  (def ab-pair-vectors (map #(vector %1 %2) a b))
  )

;returns the dot product of two n sized vectors
(defn dot-product
  [v1 v2]
  (if (nil? v1)
    0
    (+ (* (first v1) (first v2)) (dot-product (next v1) (next v2)))))

;destructures the vector of two vectors called pair-vectors
; allows the ability to send the individual vectors
; to the dot-product function
(defn pre-dot-product
  [pair-vectors]
  (let [[v1 v2] pair-vectors]
    (dot-product v1 v2)
    )
  )

;perform matrix multiplication without multi-core parallel optimization
(defn matrix-multiply
  []
  (dorun (map pre-dot-product ab-pair-vectors))
  )

;perform matrix multiplication using multi-core parallel optimization
; by using pmap with partitioned workload sets
(defn pmatrix-multiply
  []
  (dorun (pmap (fn [abpv] (doall (map pre-dot-product abpv))) (partition-all (/ m num-of-cores)  ab-pair-vectors)))
  )

;prints matrix a
(defn printa
  []
  (println "Matrix A:")
  (loop [moda a]
    (when-let [v1 (first moda)]
      (println v1)
      (recur (rest moda)))
    )
  )

;prints matrix b
(defn printb
  []
  (println "Matrix B (Transposed):")
  (loop [modb b]
    (when-let [v2 (first modb)]
      (println v2)
      (recur (rest modb))
      )
    )
  )

; main entry point of the program
(defn -main
  [& args]
  (define-constants)
  (define-matrices)
  ;(printa)
  ;(printb)
  ;(println)
  (make-multiple-pair-vectors)
  ;(print ab-pair-vectors)
  ;(println (time (matrix-multiply))) ;slower version of matrix multiplication
  (println (time (pmatrix-multiply))) ;faster, parallel matrix multiplication
  )